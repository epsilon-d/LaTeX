\documentclass{article}
\usepackage[utf8]{inputenc}
\usepackage{amsmath}
\usepackage{color}

\title{Individual Work 3}
\author{$epsilon-d$}
\date{May 2019}

\begin{document}

\maketitle

\Large 1. We are going to investigate numerical differentiation of $f(x) = \tan x$  at  $x = \frac{\pi}{4}$\\

\large(a) What is $f'(\pi/4)?$ \textcolor{blue}{[2]}

\begin{align*}
f(x) &= \tan x\\
&= \frac{\sin x}{\cos x}\\
Therefore,\\
f'(x) &= (\frac{\sin x}{\cos x})' \\ 
&= \frac{(\sin x)'\cos x - \sin x(\cos x)'}{\cos^2 x}\\
&= \frac{(\cos x)\cos x - \sin x(-\sin x)}{\cos^2 x}\\
&= \frac{\cos^2 x + \sin^2 x}{\cos^2 x}\\
&= \frac{1}{\cos^2 x}\\
&= \frac{1}{\cos^2 x}\\
f'(\pi/4) &= \frac{1}{\cos^2 \frac{\pi}{4}}\\
&=\frac{1}{(\frac{1}{\sqrt{2}})^2}\\
\end{align*}
\begin{align*}
&=\frac{1}{(\frac{1}{2})}\\
&=2\\
\end{align*}

(b) Numerically compute $f'(\pi/4)$ using forward, backward, and centered difference for step size $h = 0.1$. \textcolor{blue}{[8]}\\

f.d. : $f'(x)\approx f(\pi/4+0.1)-f(\pi/4)/0.1\approx2.23049$\par
b.d. : $f'(x)\approx f(\pi/4)-f(\pi/4-0.1)/0.1\approx1.82371$\par
c.d. : $f'(x)\approx f(\pi/4+0.1)-f(\pi/4-0.1)/0.2\approx2.0271$\\

(c) Make the following table to see error values $\epsilon$ over step size $h$. \textcolor{blue}{[10]}\\

\begin{table}[h]
\begin{tabular}{l|lll}
h             & 0.1               & 0.01              & 0.001             \\ \hline
error of f.d. & 2.230488804498650 & 2.020270043215888 & 2.002002670004388 \\
error of b.d. & 1.823711905674797 & 1.980263375464597 & 1.998002663337495 \\
error of c.d. & 2.027100355086724 & 2.000266709340242 & 2.000002666670941
\end{tabular}
\end{table}

\begin{table}[h]
\begin{tabular}{l|ll}
h             & 0.0001            & 0.00001           \\ \hline
error of f.d. & 2.000200026669452 & 2.000020000270109 \\
error of b.d. & 1.999800026663001 & 1.999980000255696 \\
error of c.d. & 2.000000026666227 & 2.000000000262903
\end{tabular}
\end{table}

\Large
2. We are going to numerically compute $\pi$ as following\\

\large
(a) Make a Matlab code to compute the value $\pi$ using the Leibniz formula \textcolor{blue}{[5]}\\

Please find the attached file.\\

(b) Prove the following formula \textcolor{blue}{[5]}\\

\begin{align*}
\pi = \int_{0}^{1} \frac{4}{1+x^2}\, dx \qquad (1)
\end{align*}

\begin{align*}
Let \; x = \tan t, \; then \\
t = \arctan x\\
If \; x = 0, \; then \; t=0 \\
If \; x = 1, \; then \; t=\frac{\pi}{4}\\
\frac{dx}{dt} &= \frac{d}{dt}\tan t \\
&= \frac{1}{\cos^2 t}dt \\
Therfore, \\
dx &= \frac{1}{\cos^2 t}dt \\
So, \int_{0}^{\frac{\pi}{4}} \frac{4}{1+x^2}\, dx &= \int_{0}^{\frac{\pi}{4}} \frac{4}{1+\frac{\sin^2 t}{\cos^2 t}}\, dt \\
&= \int_{0}^{\frac{\pi}{4}} \frac{4\cos^2 t}{(\cos^2 t+\sin^2 t)}\times\frac{1}{\cos^2 t}\, dt \\
&= \int_{0}^{\frac{\pi}{4}} \frac{4\cos^2 t}{1}\times\frac{1}{\cos^2 t} \, dt \\
&= \int_{0}^{\frac{\pi}{4}} \frac{4}{1}\, dt \\
&= \int_{0}^{\frac{\pi}{4}} 4\, dt \\
&= 4\times\frac{\pi}{4}-4\times0 \\
&= \pi
\end{align*}

(c) Make a Matlab code to compute the approximate value for $\pi$ using (1) with numerical
integration techniques; midpoint, trapezoid, and Simpson composite quadrature rules for
various step sizes $h$. \textcolor{blue}{[10]} \\

Please find the attached file.\\

(d) Make the error plot as a function of $h$. \textcolor{blue}{[10]} \\

Please find the attached file.\\

(e) Using the midpoint, trapezoid, and Simpson composite quadrature rules, compute the
approximate value for $\pi$ with 6 decimal accuracy. \textcolor{blue}{[10]} \\

Please find the attached file.\\

\normalsize Copyright 2019. $epsilon-d$ All Rights Reserved.
\end{document}